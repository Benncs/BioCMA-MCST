\documentclass{../docclass}

\usepackage{geometry}
\usepackage{hyperref}
\usepackage{enumitem}
\usepackage{fancyhdr}
\pagestyle{fancy}
\fancyhf{}
\fancyhead[L]{BioCMA-MCST CLI Documentation}
\fancyhead[R]{\today}

% Page layout
\geometry{left=1in,right=1in,top=1in,bottom=1in}

\title{BioCMA-MCST Command Line Interface (CLI) Documentation}
\author{}
\date{}

\begin{document}

\title{BioCMA-MCST CLI \\V \myversion}
\author{CASALE Benjamin}
\date{}
\maketitle
\tableofcontents 

\section*{Introduction}

The BioCMA-MCST Command Line Interface (CLI) provides essential features for running simulations in BioCMA-MCST. This CLI includes the basic and minimal features needed to execute a simulation. Please note that preprocessing and postprocessing are not directly supported through this interface and need to be executed manually.

As the CLI is designed to be generic, direct usage can be cumbersome. The CLI can be used for running batches of simulations or integrated into an automated pipeline with inter-process communication (IPC). To facilitate communication, the underlying program used by the CLI does not require user action during simulation. The user can also directly use this CLI therefore, a wrapper Python script is provided to enhance user experience. 

\subsection{Basic Command Structure}

The general syntax for using the BioCMA-MCST CLI is as follows:

\begin{verbatim}
biocma-mcst [options] [arguments]
\end{verbatim}



\appendix



\addListofFiguresTablesBibliography

\end{document}